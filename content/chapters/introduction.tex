

Germany is aligned with the digitalization goals of the European commission: by
2030, 90\% of SMEs should have reached at least a basic level of digitalization,
and 75\% of them should utilize big data applications, AI or cloud-based tools.
In addition, at least 80\% of adults should have basic or higher
digital skills \parencite{digitalization-europe-goals,
digitalization-germany-goal-alignment}. For companies, this means that more than
1\% of a company's turnover should represent its total turnover, and its
business-to-customer web sales should be more than 10\% of its total web sales
\parencite{digitalization-dii-definition}. For individuals, this means that

However, while the country has seen large investment and growth in
the last decade
(\citeneeded), a substantial amount of challenges still remain. If
Germany is to achieve all 17 sustainable goals set by the UN by 2030, there
are still a series of challenges to be overcome
\parencite{digitalization-germany-2025-status}.

\parencite{digitalization-roleInEnvironmentAndSustainability}
concluded that the implementation of digital technologies enhances
the sustainable capabilities of countries and regions by helping them
eliminate hunger, reduce poverty, reducing environmental hazards,
among others. Digitalization has the potential to be the base of
achieving all 17 sustainable development goals proposed by
\parencite{UN2015Agenda}.

\cite{report2024ImportanceOfHandworkers} highlights the importance of the crafts
sector in Germany, being the provider of jobs for 6 million people.
Additionally, it generates an approximate sum of 650 billion euros in annual
turnover. This sector represents almost 30\% of the companies in Germany.

\begin{markdown}
  We want to reach a 90\% level of digitalization by 2030.
  There are many reasons. Here they are:
  - Mention Economic benefits for companies and the EU
  - Mention Ecological benefits
  - three

  And here are the areas it entails:
  - one
  - two
  - three

  However, the current status of digitalization is not that good.
  Therefore, a lot of effort is being put into finding better ways to
  digitalize businesses.
  And it is sort of working.
  But it sort of is not, as it's quite slow, and there is a lot of
  reluctance about digitalizing businesses for these reasons:
  - one
  - two
  - three

  Therefore, we need to explore ways to support the digitalization of German
  crafts businesses.
\end{markdown}
