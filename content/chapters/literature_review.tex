
\section{Digitalization or Digitization?}

The difference between terms regarding the utilization of Digital services and
the digital world to provide or improve a company's own products or services has
been widely discussed over the years. While the most prominent terms used
to refer to these activities are "Digitalization" and "Digitization", the
difference between both has been discussed for over a decade
\parencite{2016DigitalizationVsDigitization}. Even in more recent years,
frameworks that create a distinction between these
terms regarding digital transformation have emerged
\parencite{DigitalizationOrDigitizationFramework}. While they are sometimes used
interchangeably, it is important to maintain a distinction when the usage of
either might create confusion.

While the purpose of this work is far from contributing arguments to
this discussion, it is necessary to define what each of the terms are being
utilized for. In accordance with this intention, the term Digitization
will follow the definition of \parencite{2016DigitalizationVsDigitization}: A
process where analog information, e.g. paper-based, is converted into a
representation of digital bits of 1s and 0s.

Additionally, as the objective of this work is to support the European goal of
an 80\% digital transformation level by 2030 (\citeneeded), the term
Digitalization will be utilized in agreement with the definition
provided by Eurofound (\citetitle{EuroFoundDigitalizationMeaning},
\citeyear{EuroFoundDigitalizationMeaning}): "the ongoing
integration of digital technologies and digitized data across the
economy and society."

\section{Advantages and disadvantages of digitalization}

\subsection{General advantages and disadvantages}

\parencite{digitalization-generalBenefits1-2016}

Benefits: - Improved
Competitiveness via Digital Procurement Drawbacks: - Technostress
(increasing volume of information, constant interruptions (e.g.,
e-mails, messages), and the complexity of software) - Perceived
Surveillance: Employees feel monitored by GPS tracking in vehicles
and digital timestamps. creates feeling of mistrust, increases
performance pressure.

\parencite{digitalization-generalBenefits1-2020} Benefits: -
Increased customer and employee satisfaction. Drawbacks -
Technostress due to performance monitoring and violation of privacy.

\begin{markdown}

  Benefits:
  - Expanded geographic market reach
  - Long-term survival
  - Higher quality products
  - Better customer ratings

  Drawbacks:
  - Increased operational costs (transport). This is related to expanded market
  - Low-quality providers have high risk of failure
  - Larger companies have lower ratings than solopreneurs

\end{markdown}

\subsection{Environment and sustainability}

\parencite{digitalization-sustainabilityTransition} found that
improving digital competitiveness in European SMEs fosters their
sustainability transition as well. Additionally, it also discovered
that a country's digital competitiveness directly impacts on SME's
decisions to take environmentally-positive actions. However, it also
found that small companies were less likely to adopt green strategies.

\parencite{digitalization-roleInEnvironmentAndSustainability}
concluded that the implementation of digital technologies enhances
the sustainable capabilities of countries and regions by helping them
eliminate hunger, reduce poverty, reducing environmental hazards,
among others. Digitalization has the potential to be the base of
achieving all 17 sustainable development goals proposed by
\parencite{UN2015Agenda}.

Through a systematic study,
\parencite{digitalization-sustainabilityPerformanceSMEs} found that
integrating digital technologies in SMEs brings environmental,
social, operative, economic, among other benefits. However, digital
readiness and the integration of government policies and skill
development within its
employees is necessary for a sustainable transformation.
\parencite{digitalization-businessSustainabilityEvaluationFramework},
found that proper preparation for digitalization is necessary for a
sustainable change and proposed a framework to both teach and
evaluate businesses on how to manage digitalization.

Finally, \parencite{digitalization-energyEfficiency} found that
companies in EU countries with increased levels of digitalization
have a reduced level of energy consumption and improved energy
efficiency. However, some companies with low online sales could see
their energy consumption rise. Therefore, it's necessary to take
energy-efficiency measures to properly balance this relationship.

\section{European Digitalization Status}
\subsection{Digitalization in Germany}

The current landscape of digitalization in Germany is not positive.
While in 2024 around 36\% of businesses saw themselves as pioneers in
digitalization \parencite{report2024ImportanceOfHandworkers}, that number saw a
rapid decrease. In 2025, only 32\% of companies identified themselves as
pioneers, while 64\% considered themselves technological stragglers
\parencite{2025GermanyDigitalisationReport}.

\doHighlight{Include report for 2026 when it's published}.

More specifically, digitalization in the crafts sector has seen an even slower
increase compared to other fields. In 2025, almost half of the
companies in this area have had trouble digitalizing their activities or
bringing digitalization into their work. However, despite the fact
that three out of four companies think their employees need more
digital skills, less than half actually provide training for them
\parencite{2025HandwerkDigitalisation}.

\section{PATCH Framework}

\begin{enumerate}
  \item The Problem: What are the known barriers to digital
    adoption in SMEs? (e.g., time poverty, lack of expertise).

  \item The Framework: What is DigComp 3.0? How does it define
    "competence" differently from 2.2? (Focus on the new AI/data
    literacy elements).

  \item The Method: What is Gamification/Micro-learning? Why is it
    theoretically suitable for adult learners who are busy?
\end{enumerate}

The "So What?": Based on all this reading, what are the 5 specific
requirements your app must meet to be successful? (This is the bridge
to Chapter 4).

{\LARGE \vspace{1em} Research Questions}

\begin{markdown}
  - What is the impact of guided digitalization challenges on
  digital literacy skills among craftspeople in Germany, as
  measured by DigComp 3.0?
  - Can the adoption of guided digitalization challenges be
  linked to improved digital literacy skills among craftspeople in
  Germany, as per the DigComp 3.0 framework?
  - How does the implementation of guided digitalization
  challenges influence the development of digital literacy skills
  among craftspeople in Germany, as measured by DigComp 3.0?
  - How can a web application improve adherence to the
  improvement of digital literacy for small and medium companies
  within the crafts sector in Germany?
  - How can a web app be designed through the DSRM to support the
  improvement of digital skills of craftspeople without disrupting
  their workflows?
  - What challenges does partially-guided digitalization present?
\end{markdown}

\begin{markdown}
  We want to reach an 80\% level of digitalization by 2030.
  There are many reasons. Here they are:
  - one
  - two
  - three
  And here are the areas it entails:
  - one
  - two
  - three
  However, the current status of digitalization is not that good.
  Therefore, a lot of effort is being put into finding better ways to
  digitalize businesses.
  And it is sort of working.
  But it sort of is not, as it's quite slow, and there is a lot of
  reluctance about digitalizing businesses for these reasons:
  - one
  - two
  - three

  Therefore, we need to explore how to use Digital tools to support the
  digitalization of German businesses
\end{markdown}

{\LARGE \vspace{1em} Literature Research Claims}
\begin{markdown}
  - Digitalization helps businesses increase their revenue
  - Therefore, digitalization is good for businesses.
  - Germany wants to digitalize 80\% of its businesses by 2030.
  - Germany's digitalization plan is not doing great.
  - The crafts sector is economically impactful and important for
  jobs in Germany
  - The crafts sector is one of the sectors with the lowest levels of
  digitalization
  - People in the crafts sector feel like they have no time to
  improve their level of digitalization.
  - (Bias) The literature shows a relationship between levels of
  digital literacy in a company and the level of digitalization in
  it. The higher level of literacy in its employees, the higher the
  level of digitalization the company tends to have.
  - Therefore, if we increase the level of digital literacy in a company, the
  digitalization in it will also improve.
  - The level of digital literacy can be measured by using the MDPQ
  or DigiCheck.
  - The current methods for supporting the improvement of digital literacy are
  the following:
  - (Requires research) The best method to improve the level of
  digital literacy of people, while respecting their time, is XYZ.
  (Example: through digital courses).
  - (cite required) Craftspeople would be interested in having such a tool.
  - Craftspeople are interested in digitalizing their businesses, but
  do not know how nor do they understand the tools for it. (cite required)
  - The efforts on improving digital literacy have been focused on it as an
  individual area, and only with generic courses.
  - Among the efforts explored to improve digital literacy, there have been a
  few different methods (list them)
  - DigComp 3.0 is a framework that is internationally accepted in Europe for
  describing the different skills encompassed in digital literacy.

  - There have been separate efforts into improving digital literacy and
  digitalization as two different things. However, little research has been
  done with both of them being done side-by-side
  - (Research Question) A good way to improve the level of digital literacy in a
  company is by having an app with tutorials that guide people both into
  improving their literacy and digitalizing their businesses at the same time.
  It will support their time constraints while helping them learn how to
  digitalize at the same time.
\end{markdown}

{\LARGE \vspace{1em} Methodology Claims}
\begin{markdown}
  - MDPQ is a tool that can be used to measure the level of digital literacy in
  people, and it aligns with DigComp 3.0.
  - DSRM is a tool that can be used to design a product while analyzing all the
  different perspectives of it.
  - A Design Case Study is a good way of... what?
\end{markdown}

{\LARGE \vspace{1em} Design Claims}
\begin{markdown}
  - The factors that determine success when digitalizing a business are the
  following: A, B and C. Therefore, these have to be considered when designing
  an app for this population (list them)
  - Microlearning is a perfect fit for digitalizing these companies since it
  provides both a way to do learn-while-doing while digitalizing in small
  periods, which is a great tool for companies who don't want to invest much
  time into their digitalization initially.
\end{markdown}
