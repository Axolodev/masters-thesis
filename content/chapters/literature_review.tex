
\section{Digitalization or Digitization?}

The difference between terms regarding the utilization of Digital services and
the digital world to provide or improve a company's own products or services has
been widely discussed over the years. While the most prominent terms used
to refer to these activities are "Digitalization" and "Digitization", the
difference between both has been discussed for over a decade
(\cite{2016DigitalizationVsDigitization}). Even in more recent years,
frameworks that create a distinction between these
terms regarding digital transformation have emerged
(\cite{DigitalizationOrDigitizationFramework}). While they are sometimes used
interchangeably, it is important to maintain a distinction when the usage of
either might create confusion.

While the purpose of this work is far from contributing arguments to
this discussion, it is necessary to define what each of the terms are being
utilized for. In accordance with this intention, the term Digitization
will follow the definition of (\cite{2016DigitalizationVsDigitization}): A
process where analog information, e.g. paper-based, is converted into a
representation of digital bits of 1s and 0s.

Additionally, as the objective of this work is to support the European goal of
an 80\% digitalization level by 2030 (\citeneeded), the term
Digitalization will be utilized in agreement with the definition
provided by Eurofound (\citetitle{EuroFoundDigitalizationMeaning},
\citeyear{EuroFoundDigitalizationMeaning}): "the ongoing
integration of digital
technologies and digitised data across the economy and society."

\section{European Digitalization Status}

\subsection{Digitalization in Germany}

The current landscape of digitalization in Germany is not positive.
While in 2024 around 36\% of businesses saw themselves as pioneers in
digitalization (\cite{report2024ImportanceOfHandworkers}), that number saw a
rapid decrease. In 2025, only 32\% of companies identified themselves as
pioneers, while 64\% considered themselves technological stragglers
(\cite{2025GermanyDigitalisationReport}).

\doHighlight{Include report for 2026 when it's published}.

More specifically, digitalization in the crafts sector has seen an even slower
increase compared to other fields. In 2025, almost half of the
companies in this area have had trouble digitalizing their activities or
bringing digitalization into their work. However, despite the fact
that three out of four companies think their employees need more
digital skills, less than half actually provide training for them
(\cite{2025HandwerkDigitalisation}).

\begin{enumerate}
  \item The Problem: What are the known barriers to digital
    adoption in SMEs? (e.g., time poverty, lack of expertise).

  \item The Framework: What is DigComp 3.0? How does it define
    "competence" differently from 2.2? (Focus on the new AI/data
    literacy elements).

  \item The Method: What is Gamification/Micro-learning? Why is it
    theoretically suitable for adult learners who are busy?
\end{enumerate}

The "So What?": Based on all this reading, what are the 5 specific
requirements your app must meet to be successful? (This is the bridge
to Chapter 4).
