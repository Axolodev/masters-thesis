\cite{dsr-three-modes} proponen 3 modos de hacer DSR combinado con
HCI dependiendo del tipo de sistema que se analiza: interno, externo
y Gestalt. El modo interno se centra en la creación de algoritmos o
artefactos computacionales. El modo externo se centra en entender
mejor los modos humanos de funcionamiento al interactuar con un
sistema. El modo Gestalt hace combinación de ambos. En este caso, me
enfocaré en hacer un análisis externo.

The Mobile Device Proficiency Questionnaire was originally created as
a tool to evaluate the digitalization level of older audiences
(\cite{mobile-device-proficiency}). It has, however, been also
utilized to evaluate the digitalization level of other audiences (\citeneeded)

{\LARGE \vspace{1em} Possible Research Questions}

\begin{itemize}
  \item What is the impact of guided digitalization challenges on
    digital literacy
    skills among craftspeople in Germany, as measured by DigComp 3.0?
\end{itemize}

Methodology is the method that I'm using to find a gap and design my experiment.
2 RQs and 3 discussions points

\clearpage
