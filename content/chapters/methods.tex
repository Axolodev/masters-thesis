\cite{dsr-three-modes} proponen 3 modos de hacer DSR combinado con
HCI dependiendo
del tipo de sistema que se analiza: interno, externo y Gestalt. El modo interno
se centra en la creación de algoritmos o artefactos computacionales. El modo
externo se centra en entender mejor los modos humanos de funcionamiento al
interactuar con un sistema. El modo Gestalt hace combinación de ambos. En este
caso, me enfocaré en hacer un análisis externo.

\

{\LARGE \vspace{1em} Possible Research Questions}

\begin{itemize}
  \item How does the use of a preexisting digitalization framework
    (PATCH) link to
    improving the digital capabilities in craftspeople?
  \item How does the use of a tool for guided digitalization support
    the improvement of
    digital literacy in craftspeople who are business owners?
  \item Can we improve the level of digital literacy in digitally illiterate
    craftspeople by guiding them through a series of digitalization challenges?
  \item What is the impact of guided digitalization challenges on
    digital literacy
    skills among craftspeople in Germany, as measured by DigComp 3.0?
\end{itemize}

Methodology is the method that am using to find a gap and design your
experiment.

2 RQs and 3 discussions points

\clearpage
