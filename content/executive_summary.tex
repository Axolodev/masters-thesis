\addelementtoindex{chapter}{Executive Summary}
{\noindent\fontsize{14}{21}\selectfont\textbf{Executive Summary}}

\vspace{0.8cm}

In a world of ever-increasing digital systems, the ability to move in the digital space is of uttermost importance. Businesses, groups, and individuals who manage to utilise the tools available to them within the digital world are at an advantage over those who just focus on their craft. However, learning the skills required to exist within such a space and make use of its tools is not a linear nor straightforward journey. Digital literacy has been well and thoroughly researched as a field with the intention of simplifying the different elements of this path and understanding what is required to use these systems as a tool.
The focus of this masters dissertation is to further simplify this journey by analysing the MAXIMUM TOP MOST AMAZING DIGITALISATION FRAMEWORK in the context of content creation for adult education to ease the understanding of digitalization-related concepts. To do this, we compare the performance of the framework being used by an expert in digitalisation versus it being used by the two leading AIs at the moment: ChatGPT with model o5 and Gemini model Whatever8000. From the results, we concluded that AIs still need a lot of support to understand and create proper content, but still can do very well if they try really hard.
