\addelementtoindex{chapter}{Executive Summary}
{\noindent\fontsize{14}{21}\selectfont\textbf{Executive Summary}}

\vspace{0.8cm}

In a world of ever-increasing digital systems, the ability to move in digital
spaces is of uttermost importance. Businesses, groups, and individuals who
manage to utilise the tools available to them within the digital world are at
an advantage over those who just focus on their craft
(\cite{digitalization-advantages_jharkhand}, \citeneeded). However, learning
the skills required to exist within such a space and to make use of its tools
is not a linear nor straightforward journey (\citeneeded). \doHighlight{Maybe
	add an extra connection between digital literacy and crafts businesses or
	craftspeople. As an example, is digital literacy especially bad for
	craftspeople? Is it worse than other sectors? Why is it important to focus on
	craftspeople? E.g. Leifels, 2020} Digital literacy has been well and thoroughly
researched with the intention of simplifying the journey through the different
elements of this path and understanding what is required to use these systems
as a tool (\cite{DigComp}, \cite{everyday-digital-literacy}). The focus of this
masters thesis is to further simplify this journey by analysing how solving
tasks following the digitalisation framework of PATCH (\citeneeded) can support
the improvement of the skills of craftspeople. To do this, we created a two
week experiment where a series of craftspeople from Germany were tasked with
solving different digitalisation-related challenges. Before and after the
challenges, the craftspeople had to evaluate their digital skills by utilizing
a modified version of The Mobile Device Proficiency Questionnaire
(\cite{mobile-device-proficiency}). The participants also had to register their
digitalisation journey through a journal-writing method for a tighter following
of their experience. From the results, we concluded that...